l%----------------------------------------------------------------------------------------
%   PACKAGES AND OTHER DOCUMENT CONFIGURATIONS
%----------------------------------------------------------------------------------------
\documentclass[10pt,a4paper]{article}
\usepackage{geometry}
\geometry{
    a4paper,
    left=10mm,
    right=10mm,
    headsep=1mm,
%     % tmargin=0pt,
    top=19mm,
    headheight=33pt,
    bottom=25mm
%     % voffset=0pt,
%     % footskip=20pt,
% includeheadfoot
}
% \addtolength{\topmargin}{2cm}
% \PassOptionsToPackage
% \geometry{article, margin=1cm}
\usepackage{marginnote} % Required for margin notes
\usepackage{datetime}
\usepackage{adjustbox}
\usepackage{pgfplotstable}
\usepackage{pgfplots}
\pgfplotsset{compat=1.11}
\usepackage{xcolor}
\usepackage{array}
\usepackage{csvsimple}
\usepackage{tcolorbox}
\usepackage{placeins}
\usepackage{hyperref} % makes contents with hyperrefferences
\usepackage{wallpaper} % Required to set each page to have a background
\usepackage{lastpage} % Required to print the total number of pages
% \usepackage[left=1.5cm,right=1.5cm,top=2.4cm,bottom=4.0cm,marginparwidth=0cm, headsep=1.5cm]{geometry}
\usepackage{amsmath} % Required for equation customization
\usepackage{amssymb} % Required to include mathematical symbols
\usepackage{tabularx}%
\usepackage{xcolor, colortbl} % Required to specify colors by name
\usepackage{listliketab}
\usepackage[utf8]{inputenc}
%\usepackage{fourier}
\usepackage{makecell}
\usepackage{booktabs}
\usepackage{siunitx}
\usepackage{graphicx}
\usepackage{enumitem}
\usepackage{sidenotes}
\usepackage{caption}
\usepackage{ragged2e}
\usepackage{needspace}
\usepackage{geometry}




\renewcommand\theadalign{b}
\renewcommand\theadfont{\normalsize}
%\renewcommand{\thefootnote}{\roman{footnote}}

% Set Font settings:
\usepackage{arev} % sans-serif math font
\usepackage{helvet} % sans-serif text font
\renewcommand{\rmdefault}{\sfdefault}
\usepackage[math]{blindtext}

\newenvironment{leftside}
  {\begin{flushleft}}
  {\end{flushleft}}


% ----------Set arial as typo ------
% \renewcommand{\rmdefault}{phv} % Arial
% \renewcommand{\sfdefault}{phv} % Arial
\newcommand{\rad}{\text{rad}}

% ---------define colors ------------
% ---------define colors ------------
\definecolor{blue}{rgb}{0, 0.68, 0.94}
\definecolor{mygray}{gray}{0.6}
\definecolor{ao}{rgb}{0.0, 0.0, 1.0}
\definecolor{tbsyellow}{rgb}{0.94, 0.58, 0.039}  
\definecolor{headerblue}{RGB}{170, 170, 170}
\definecolor{rowgray}{gray}{0.95}
\definecolor{staco-red}{HTML}{ee1d24}
\definecolor{Red}{rgb}{215, 67, 21}
\definecolor{okgreen}{RGB}{220,245,220}
\definecolor{badred}{RGB}{245,220,220}
% ---------------- Toggles ----------------
\providetoggle{oneway_ok}
\providetoggle{twoway_ok}
\providetoggle{moment_ok}
\providetoggle{dev_ok}
\providetoggle{bearing_ok}

% Python will set these:
% \settoggle{oneway_ok}{true}
% \settoggle{twoway_ok}{false}



%------------------subsection indent-------------
\usepackage{changepage,lipsum,titlesec}% http://ctan.org/pkg/{changepage,lipsum,titlesec}
\titleformat{\section}[block]{\bfseries}{\thesection.}{1em}{}
\titleformat{\subsection}[block]{}{\thesubsection}{1em}{}
\titleformat{\subsubsection}[block]{}{\thesubsubsection}{1em}{}
\titlespacing*{\subsection} {2em}{3.25ex plus 1ex minus .2ex}{1.5ex plus .2ex}
\titlespacing*{\subsubsection} {3em}{3.25ex plus 1ex minus .2ex}{1.5ex plus .2ex}


%------------------------------------------------
\usepackage{fancyhdr} % Required to customize headers
% \setlength{\headheight}{2pt} % Increase the size of the header to accommodate meta-information
% \setlength{\footerheight}{2pt}
\pagestyle{fancy}\fancyhf{} % Use the custom header specified below
\renewcommand{\headrulewidth}{0pt} % Remove the default horizontal rule under the header

\setlength{\parindent}{0cm} % Remove paragraph indentation
\setlength{\unitlength}{1mm} % Set the unit length to millimeters

\begin{document}
%----------------------------------------------------------------------------------------
%   HEADER & FOOTER INFORMATION
%----------------------------------------------------------------------------------------
\
\fancyhead[R]{ \includegraphics[height=1cm, width=\textwidth, keepaspectratio]{di.png}}

\fancyfoot[L]{\begin{tcolorbox}[width=\textwidth,top=1pt, bottom=1pt, colframe=staco-red, colback=staco-red , arc=0.5mm]\end{tcolorbox}\begin{center} | www.designintelligence.tech | ISOLATED FOOTING DESIGN REPORT\:\:| Page \thepage
    \end{center}}

%----------------------------------------------------------------------------------------
%   MAIN CONTENT
%----------------------------------------------------------------------------------------

    % Insert title block
    \begin{tcolorbox}[width=\textwidth,top=1pt, bottom=1pt, colframe=staco-red, colback=staco-red, arc=0.5mm]
        \color{white}\hspace{0.14cm}{\large \textbf{Calculation of Isolated Footing Design}\hfill C-14-2.0d}
    \end{tcolorbox}

    % Insert table with general information
    \vspace{0.3cm}\hspace{0.5cm}
    \begin{tabularx}{\textwidth}{@{}m{20mm} @{}m{50mm}@{}m{25mm} @{}m{50mm}}
        \textbf{Company:} &
        {Design Intelligence LLP}
        & \textbf{Project:} &
        {Isolated Footing Designer}
        \\
        \textbf{Author:} &
        {Design Intelligence LLP}
        & \textbf{Project code:} &
        {C-14-2.0d}
        \\
        \textbf{Date:} &
        {\today}
        &&
    \end{tabularx}
    \vspace{0.3cm}\hspace{0.7cm}
    \begin{tcolorbox}[width=\textwidth,top=1pt, bottom=1pt, colframe=staco-red, colback=staco-red, arc=0.5mm]
\color{white}\hspace{0.14cm}{\large \textbf{General Information about the Project}}
\end{tcolorbox}
\begin{minipage}[l]{0.5\textwidth}
\vspace{0pt}
\centering
\includegraphics[scale = 0.4, keepaspectratio]{footing.png}
\end{minipage}
\begin{minipage}[r]{0.5\textwidth}
\vspace{5pt}
An Isolated Footing is a type of Shallow foundation used to support a single, individual column or pillar, commonly in residential or low-rise buildings. It acts as a pad—typically square or rectangular—that spreads concentrated loads from the column onto the soil, making it a cost-effective, easy-to-construct, and widely used foundation choice when soil bearing capacity is high.
\vspace{-\baselineskip}
\vspace{0.3cm}\hspace{0.6cm}
\end{minipage}
\section*{Input Parameters:}
 \noindent\color{lightgray!50}\rule{\linewidth}{1pt}
    % \newline
    % \newline
    \color{black}
        \hspace{0.6cm}
        \begin{tabularx}{\textwidth}{@{}m{70mm} @{}m{40mm} @{}m{25mm} @{}m{10mm} @{}m{15mm} @{}m{20mm}}
                                        \bfseries{Strength Properties} & \multicolumn{5}{r}{} \\
                                        \vspace{10pt}
                                        {Concrete Grade}:                        & {M-20}  & $f_{ck}$ &  $20$ & $N/mm^2$   \\
                                        \vspace{10pt}
                                        Reinforcement Grade:                            & {Fe-415} & $f_y$ & $415$ & $N/mm^2$  \\
                                          \vspace{0.3cm}\hspace{0.6cm}
    \end{tabularx}
    \noindent\color{lightgray!50}
    % \newline
    % \newline
    \color{black}
        \hspace{0.6cm}
    \begin{tabularx}{\textwidth}{@{}m{70mm} @{}m{40mm} @{}m{15mm} @{}m{10mm} @{}m{15mm} @{}m{20mm}}
                                        \bfseries{Column Data Input} & \multicolumn{5}{r}{} \\
                                        \vspace{10pt}
                                        Column Width      & $a$   & = & & {$0.3$}  & m   \\
                                        \vspace{10pt}
                                       Column Length:       & $b$   & = & & {$0.5$}  & m   \\
                                       \vspace{10pt}
                                       \bfseries{Footing Data Input} & \multicolumn{5}{r}{} \\
                                       \vspace{10pt}
                                        Footing Width       & $X$   & = & & {$1.8$} & m   \\
                                        \vspace{10pt}
                                       Footing Length:       & $Y$   & = & & {$3.6$} & m   \\ 
                                       \vspace{10pt}
                                        Footing Depth:       & $d$   & = & & {$692$} & mm   \\ 
                                        \vspace{10pt}
                                       \bfseries{External Factors} & \multicolumn{5}{r}{} \\
                                       \vspace{10pt}
                                       Safe Bearing Capacity:       & $q_u$   & = & & {$200$} & kN/m$^2$   \\
                                       \vspace{10pt}
                                       External Moment:       & $m_u$   & = & & {$540$} & kN-m   \\
                                       \vspace{10pt}
                                       Column Load:       & $W_u$   & = & & {$1150$} & kN   \\
                                       
                                       
                                        
\end{tabularx}
\newpage
 \begin{tcolorbox}[width=\textwidth,top=1pt, bottom=1pt, colframe=staco-red, colback=staco-red, arc=0.5mm]
        \color{white}\hspace{0.14cm}{\large \textbf{Deign Assumptions}}
    \end{tcolorbox}
    \vspace{2mm}
\section*{Design Assumptions}
\begin{minipage}[l]{0.5\linewidth}
\begin{tcolorbox}[colback=rowgray,colframe=black]
\begin{itemize}
  \item Column is axially loaded.
  \item Footing is rectangular in shape.
  \item Footing rests on homogenous soil.
  \item Self-weight of footing is assumed as 10\% of column load, i.e.$$W_f = 10\% * W_u$$.
  \item Design is carried out as per IS 456:2000.
\end{itemize}
\end{tcolorbox}
\end{minipage}
\hspace{0.6cm}
\vspace{-\baselineskip}
\begin{minipage}[c]{0.47\textwidth}
    \centering
    \includegraphics[width=0.8\linewidth]{footing2.png}
    \captionof{figure}{This figure is according to input values}
    \label{fig:placeholder}
\end{minipage}
\vspace{3cm}
 \begin{tcolorbox}[width=\textwidth,top=1pt, bottom=1pt, colframe=staco-red, colback=staco-red, arc=0.5mm]
        \color{white}\hspace{0.14cm}{\large \textbf{Bearing Pressure Check according to IS 456: 2000}}
    \end{tcolorbox}
\begin{adjustwidth}{5mm}{0mm}

\textbf{Total Load:}

\[
W = W_u + W_f
\]

\[
W = 1150 + 115 = 1265 \ \text{kN}
\]

\textbf{hence,}
\[
W = 1265 \ \text{kN}
\]

\textbf{Soil Pressure:}
\[
p_u = \frac{W_u + W_f}{X \times Y}
     = \frac{1265}{3.6 \times 1.8}
     = 195.06 \ \text{kN/m}^2
\]

\textbf{Factored Soil Pressure:}
\[
p_{u,f} = 1.5 \, p_u
        = 1.5 \times 195.06
        = 292.82 \ \text{kN/m}^2
\]

\textbf{Since } $p_u \le q_u$,

{\colorbox{okgreen}{SAFE}}

\normalsize
\end{adjustwidth}

\hspace{0.6cm} \section*{Conclusion}

\begin{adjustwidth}{5mm}{0mm}
The footing is safe in bearing against the applied load.
\end{adjustwidth}

\clearpage

\begin{tcolorbox}[width=\textwidth,top=1pt, bottom=1pt, colframe=staco-red, colback=staco-red, arc=0.5mm]
        \color{white}{\large \textbf{One-way Shear Check according to IS 456: 2000}}
    \end{tcolorbox}

\noindent {\hspace{0.2cm}\normalsize{ \textbf{One-Way Shear Check}}}
(IS 456:2000, Cl. 31.6.1)


\normalsize{Critical section for one-way shear is located at a distance 
$d$\,from the face of the column.}

\vspace{2mm}

\textbf{Shear force in shorter direction:}

\[
V_{u,x} = p_u \,\times Y\times
\left(
\frac{X}{2} - \frac{a}{2} - d
\right)
\]

\[
V_{u,x} = 195.06 \times 3.6
(0.9 - 0.15 - 0.692)
= 40.8 \ \text{kN}
\]

\textbf{Shear force in longer direction:}

\[
V_{u,y} = p_u \,\times X\times
\left(
\frac{Y}{2} - \frac{b}{2} - d
\right)
\]

\[
V_{u,y} = 195.06 \times 1.8
(1.8 - 0.30 - 0.692)
= 283.6 \ \text{kN}
\]

\textbf{Governing shear force:}

\[
V_u = \max(V_{u,x}, V_{u,y})
= \max(40.8, 283.6)
= 283.6 \ \text{kN}
\]

\textbf{Nominal shear stress:}

\[
\tau_v = \frac{V_u}{b \, d}
= \frac{283.6}{1.8 \times 0.692}
= 0.23 \ \text{N/mm}^2
\]

Permissible shear stress (Table 19, IS 456):

\[
\tau_c = 0.36 \ \text{N/mm}^2
\]

\textbf{Since } $\tau_v \le \tau_c$,

\vspace{2mm}
\colorbox{okgreen}{\textbf{SAFE}}

\normalsize
    \centering
    \includegraphics[width=0.9\linewidth]{oneway.png}
    \label{fig:placeholder}
\begin{tcolorbox}[width=\textwidth,top=1pt, bottom=1pt, colframe=staco-red, colback=staco-red, arc=0.5mm]
        \color{white}\hspace{0.14cm}{\large \textbf{Two-way Shear Check according to IS 456: 2000}}
        \end{tcolorbox}

\begin{adjustwidth}{0.14cm}{0cm}
\begin{minipage}[l]{0.5\linewidth}
\justifying
\normalsize{Critical perimeter at $d/2$ from column face:}
\[
b_0 = 2[(a + d) + (b + d)]
     = 2[(0.3 + 0.692) + (0.6 + 0.692)]
     = 4.57 \ \text{m}
\]
Punching shear force:
\[
V_u = p_u [XY - (a+d)(b+d)]
\]
\[
V_u = 195.06 [1.8 \times 3.6 - 0.992 \times 1.292]
     = 1015.2 \ \text{kN}
\]
Punching shear stress:
\[
\tau_v = \frac{V_u}{b_0 \, d}
       = \frac{1015.2}{4.57 \times 0.692}
       = 0.32 \ \text{N/mm}^2
\]
Permissible punching shear stress:
\[
\tau_c = 0.25 \sqrt{f_{ck}}
       = 1.12 \ \text{N/mm}^2
\]
\textbf{Since } $\tau_v \le \tau_c$,
\vspace{2mm}
\colorbox{okgreen}{\textbf{SAFE}}
\normalsize
\end{minipage}
\hfill
\begin{minipage}[c]{0.45\linewidth}
\centering
\includegraphics[width=0.68\linewidth]{twoway.png}
\captionof{figure}{Punching shear critical perimeter}
\end{minipage}

\end{adjustwidth}
\begin{tcolorbox}[width=\textwidth,top=1pt, bottom=1pt, colframe=staco-red, colback=staco-red, arc=0.5mm]
        \color{white}\hspace{0.14cm}{\large \textbf{Moment Check according to IS 456: 2000}}
        \end{tcolorbox}

\begin{adjustwidth}{5mm}{0mm}
Moment of Resistance:
assuming ($Xumax / d) = 0.48$ 
\[
M_r = 0.138 \, f_{ck} \, b \, d^2
\]
\[
M_r = 0.138 \times 20 \times 1800 \times 692^2 \times 10^{-6}
     = 475 \ \text{kN·m}
\]
Applied moment:
\[
M_u = 240 \ \text{kN·m}
\]
\textbf{Since } $M_u \le M_r$,
\vspace{2mm}
\colorbox{okgreen}{\textbf{SAFE}}
\normalsize
\end{adjustwidth}
 \centering
    \includegraphics[width=0.85\linewidth]{moment.png}
    \captionof{figure}{Moment generation}
    \label{fig:placeholder}


\begin{tcolorbox}[width=\textwidth,top=1pt, bottom=1pt, colframe=staco-red, colback=staco-red, arc=0.5mm]
        \color{white}\hspace{0.14cm}{\large \textbf{Reinforcement Detailing and Spacing Calculation according to IS 456: 2000}}
        \end{tcolorbox}

\begin{adjustwidth}{0.14cm}{0cm}
\justifying
\fontsize{9}{11}\selectfont

\textbf{Required steel area (IS 456:2000, Cl. 26.5.2)}

\vspace{2mm}

\textbf{Required reinforcement from bending moment:}

\[
A_{st,req} = \frac{M_u}{0.87 f_y \, j \, d}
           = \frac{240 \times 10^6}
           {0.87 \times 415 \times 0.9 \times 692}
           = 1045 \ \text{mm}^2
\]

\textbf{Minimum reinforcement as per IS 456:2000:}

\[
A_{st,min} = 0.12\% \times b \times D
           = 0.0012 \times 1800 \times 692
           = 1494 \ \text{mm}^2
\]

\textbf{Governing reinforcement:}

\[
A_{st,prov} = \max(A_{st,req}, A_{st,min})
            = 1494 \ \text{mm}^2
\]

\textbf{Since } $A_{st,prov} \ge A_{st,min}$,
\quad
\colorbox{okgreen}{\textbf{SAFE}}

\vspace{2mm}

\textbf{Assumed reinforcement:} $\phi 16$ mm bars

\textbf{Area of one bar:}

\[
A_{bar} = \frac{\pi}{4} \times 16^2
        = 201 \ \text{mm}^2
\]

\textbf{Spacing of reinforcement bars:}

\[
s = \frac{1800 \times A_{bar}}{A_{st,prov}}
  = \frac{1800 \times 201}{1494}
  = 242.16 \ \text{mm}
\]

\textbf{Provide:}
\quad
$\phi 16$ mm bars @ 240 mm c/c (both directions)

\vspace{1mm}
\colorbox{okgreen}{\textbf{REINFORCEMENT ADEQUATE}}
\vspace{1cm}

\includegraphics[width=0.5\linewidth]{bar.png}
\captionof{figure}{Reinforcement detailing}
\label{figure:placeholder}

\end{adjustwidth}

\clearpage

\begin{tcolorbox}[width=\textwidth,top=1pt, bottom=1pt, colframe=staco-red, colback=staco-red, arc=0.5mm]
        \color{white}\hspace{0.14cm}{\large \textbf{Development Check according to IS 456: 2000}}
        \end{tcolorbox}
\begin{adjustwidth}{5mm}{0mm}
\textbf{Development Length Check}\, according to IS 456:2000, Cl. 26.2.1
\normalsize

\[
L_d = \frac{0.87 f_y \phi}{4 \tau_{bd}}
\]

where $\tau_{bd}$ = Bond stress
\vspace{1cm}

For Fe 415:

\[
L_d = 47 \phi = 47 \times 16 = 752 \ \text{mm}
\]

Available length:

\[
L_{avail} = \frac{X - a}{2}
          = \frac{1.8 - 0.3}{2}
          = 750 \ \text{mm}
\]

\textbf{Since } $L_{avail} \le L_d$,



\vspace{2mm}
\colorbox{badred}{\textbf{DEVELOPMENT LENGTH NOT ADEQUATE}}\,
\textbf{Revise footing or Provide hooks}

\normalsize
\end{adjustwidth}
\vspace{1cm}

\large
\textbf{\underline{Design Summary:}}
\vspace{0.5cm}
\begin{adjustwidth}{5mm}{0mm}
\begin{itemize}
\large
  \item Footing size: $1.8 \times 3.6$ m
  \item Effective depth: 692 mm
  \item Bearing pressure: SAFE
  \item One-way shear: SAFE
  \item Punching shear: SAFE
  \item Bending moment: SAFE
  \item Reinforcement: 16 mm @ 240 mm c/c
\end{itemize}
\end{adjustwidth}

        
\end{document}
